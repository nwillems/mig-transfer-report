\documentclass[paper=a4, fontsize=11pt]{scrartcl} % A4 paper and 11pt font size

\usepackage[T1]{fontenc} % Use 8-bit encoding that has 256 glyphs
\usepackage{fourier} % Use the Adobe Utopia font for the document - comment this line to return to the LaTeX default
\usepackage[english]{babel} % English language/hyphenation
\usepackage{amsmath,amsfonts,amsthm} % Math packages
\usepackage{hyperref}

\usepackage{lipsum} % Used for inserting dummy 'Lorem ipsum' text into the template

\usepackage{sectsty} % Allows customizing section commands
\allsectionsfont{\centering \normalfont\scshape} % Make all sections centered, the default font and small caps

\usepackage{fancyhdr} % Custom headers and footers
\pagestyle{fancyplain} % Makes all pages in the document conform to the custom headers and footers
\fancyhead{} % No page header - if you want one, create it in the same way as the footers below
\fancyfoot[L]{} % Empty left footer
\fancyfoot[C]{} % Empty center footer
\fancyfoot[R]{\thepage} % Page numbering for right footer
\renewcommand{\headrulewidth}{0pt} % Remove header underlines
\renewcommand{\footrulewidth}{0pt} % Remove footer underlines
\setlength{\headheight}{13.6pt} % Customize the height of the header

\numberwithin{equation}{section} % Number equations within sections (i.e. 1.1, 1.2, 2.1, 2.2 instead of 1, 2, 3, 4)
\numberwithin{figure}{section} % Number figures within sections (i.e. 1.1, 1.2, 2.1, 2.2 instead of 1, 2, 3, 4)
\numberwithin{table}{section} % Number tables within sections (i.e. 1.1, 1.2, 2.1, 2.2 instead of 1, 2, 3, 4)

\setlength\parindent{0pt} % Removes all indentation from paragraphs - comment this line for an assignment with lots of text

\begin{document}

\section{Project title}
Proposed: ``Transferring large data quantities to MiG storage resources''

\section{Problem statement}
Is it possible to develop a system, that out-of-band\footnote{Out-of-band with 
respect to the originating entity} synchronises data to a vgrid storage 
resource from an external location, while satisfying the principles of Minimum 
intrusion Grid?

\section{Motivation}
Why is this project interesting. Basically a more elaborate problem description

When research data is recorded at a research station, it needs to be
transferred to MiG. Today the data is mainly coming from neutron pictures
captured in Switzerland and Germany, these images are in the terabyte 
size-order. The main way of data transfer is by plane or train, stored on
external hard-drives, and are then transferred ``locally'' to MiG.

The manual transfer has several issues:
\begin{itemize}
\item Cost - 3TB by copper is virtually free, where as by plane can be quite
    costly.
\item Data corruption - It is hard to know, if a hard-drive corrupts data while 
    travelling.
\item Time - The process of transferring data from drive to PC to MiG, needs
    supervision, thereby taking away time from other tasks.
\end{itemize}

The project shall concern developing a solution to resolve these issues, while
still maintaining the principles of MiG\cite{cpa2005arch}. The below list 
highlights some of the principles behind MiG:

\begin{description}
\item [Security] The current use of certificates, should provide the basis for
    identity management and transport-layer security
\item [Non-intrusive] The user should not be concerned with the actual
    transferring of data
\item[Fault tolerant] Data transfers are supposed to tolerate common problems
    involved when working with the internet(Connection loss, High ping times,
    etc.)
\item[Scalable] The overall speed of the system should be dependent on data 
    size, and possibly amount of transfers
\end{description}

\section{Learning goals}
Measurable goals that should layout the framework for tasks.

\begin{itemize}
\item Integrate with current MiG infrastructure
\item Reflect and demonstrate security concerns with respect to the complete
    system
\item Demonstrate correctness of the developed system
\item Explain X.509 structure and principles, with respect to the current
    use in MiG
\item Understand and reason about the transferring of data between entities
\end{itemize}

\section{Project limits - not completed}
The project will exclusively investigate a scenario wherein a  hardware-based
entity, controlled by the MiG-core team, is empowered to upload on behalf of
a user. Furthermore the project will not investigate its own data-transfer
protocols, but exclusively consider existing means of data-transfer.

\section{Tasks - 80\% done}
The below specified tasks are a rough sketch of what parts the project
involves, they are mainly concerned with the report. The development process is
expected to be done in parallel with the relevant report sections, but with a
hint of iterative development.

\begin{description}
\item[Task 0] Write synopsis
\item[Milestone 0] Hand-in synopsis(\textbf{23/09/2013})
\item[Task 1] Research, gather and read materials
\item[Task 2] Write introductory sections about the problem area and eco-system
    surrounding it
\item[Task 3] Document work and decisions together with new discoveries
\item[Milestone 1] Hand-in midway report(\textbf{18/11/2013})
\item[Task 3 - continued]
\item[Task 4] Write parts/sections about the actual implemented system
\item[Task 5] Finalise report sections
\item[Milestone 2] Hand-in final report(\textbf{10/01/2014})
\end{description}


\section{Below sections not done}

\paragraph{Literature}
Probably reference to some of the RFC' covering X.509, articles about MiG,
something about transferring data, and more...


\bibliographystyle{plain}
\bibliography{literature}

\end{document}
